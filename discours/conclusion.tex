\section{Conclusion et Synthèse}


\subsection{Q1. }



\subsection{Conclusion}

\paragraph{}
On a proposé d'assister le developppement évolutionnaire des SdSs par
de la reconfiguraiton dynamique. 

\paragraph{}
Pour cela on a d'abord montré que 
-> les documentations de l'état de
l'art était limité, 
-> on montrer le caractère artisanal dans la conception des
reconfigurations 

\paragraph{} 
On a proposé une expérimentation 
-> developpé un framework de reconfiguration simulant l'exection d'un
SdS et de reconfiguration dynamique. 
-> On a modélisé un cas d'étude comportant des reconfigurations à
réaliser
-> Cela nous a permi de montrer le caractère réutilisable de notre
documentation 
-> mais aussi l'utilité de processus de reconfiguration  pour assister
l'architecti en le guidant et l'aide à maîtriser la dégradation de
service pendant la reconfiguration.  


\subsection{Perspectives}

\paragraph{} 
a court terme on propose d'améliorer l'expérimentation.
L'expérimentation est limité dans la mesure ou les sujets ne sont pas
externent au travaille de conception des patrons de reconfiguration et
du processus. 
Ensuite l'étude de cas n'est pas complète. 

On proposerait d'améliorer la modélisation de l'étude de cas en
impliquant les systèmes consituants réèlle. 

Ensuite, on souhaiterais pouvoir réaliser deux populations ... 


\paragraph{}
Ensuite on souhaiterai intégrer dans notre environnement de
reconfiguration un outil pour permet le suivi des décisions de
reconfiguration. On envisage par exemple, la possibilité d'annoter un
ensemble d'opération de reconfiguration correspond à une décision de
reconfiguration. 

\paragraph{} 
On pourrait envisager ensuite l'étude de la composabatilté des
patrons de reconfiguration. Sur la base de la fréquence d'utilisation
des patrons de reconfigurations conjointe pour pourrait en déduire des
catalogue de patron de reconfiguration pour des domaitnes en
particluer. 


\paragraph{} 
A ce stade, la sûreté, c'est à dire la garantie qu'ils vont réaliser
ce qu'on attend, tiens sur le fait que ses patrons sont produits à
partir des expériences tirées de leur mise en \oe{}vre.

Les méthodes de vérification sont couteuse en temps de conception ou
de calcule. Le caractère réutilisable des patrons sont p-être une solution pour compenser le
coût. Des méthodes de vérification que l'on envisage sont par exemple
la preuve de théorème. peut impliquer des preuvent des grandes donc
couteuse en ressource humaine. les méthodes de modèle checking pose
habituellement des problèmes de temps de calcule. 


\paragraph{} 
Pour finir, dans cette thèse on pas pris en compte la spécificité des
systèmes intégrant des être humains qui est une caractéristique des
SdS. La queston qu'on ce pose est comment intégrer dans notre
documentation les contraintes liées aux aspects sociaux propre. 
devrait intégrer une description sociale impliqué dans la conduite des
changement 


 
