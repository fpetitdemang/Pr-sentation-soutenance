\section{Expérimentation}

\subsection{Description de l'expérimentation}

\paragraph{}
Pour vérifier la fesabilité de notre approche on a proposé une
expérimentation. 

\paragraph{}
Pour vérifier la fesabilité de notre approche on a proposé une
Pour cela, on a développé un framework de
reconfiguration dans le but de simuler l'execution d'une sds
reconfiguré. Le framework embarque des opérations de reconfiguration
et mécanismes de reconfiguration à disposition de l'architecte pour
décrire le script. 
Ensuite le framework peut être configuré avec l'architecture à
vérifier pendant la reconfiguration et la description de scénario
d'execution. 

\paragraph{}
On a fourni la description d'une architecture de SdS
et d'une évolution à déployer. Pour cela on a modélisé deux
configurations de l'étude de cas que nous avons étudié dans cette
thèse. L'objectif de la reconfiguration était de déléguer le
commandement d'une ensemble de système constituant. Les propriétés
vérifiées étaient que l'ensemble des opérateurs restent sous la
supervision d'un organe de contrôle et qu'à la fin de la
reconfiguration l'état de l'avancement de la mission ne soit pas
perdu. 

\paragraph{}
Pour vérifier la fesabilité de notre approche on a proposé une
Pour aider l'architecte de la reconfiguration nous avons fourni
l'ensemble des patrons de reconfiguration dynamique ainsi que la
description de processus de reconfiguration dévéloppé dans la thèse. 

\subsection{Interprétation et limite}

\paragraph{}
Pour réalisé la reconfiguration, l'architecte a réalisé trois
itérations en utilisant le patron de co-évolution et de tranquilité. 
L'execution de la reconfiguration a respecté les objectifs de la
reconfiguration. 

\paragraph{} 
On a essayé d'évaluer la réutilisation des patrons de reconfiguration
en vérifiant que l'architecte les réutilisent dans la conception de
son script. 
On s'est posé la question de savoir si l'architecte à pu envisagé
plusieurs solution de reconfiguration, choisir la plus adapté à
l'objectif de reconfiguration et à réussie à l'implémenté. 


\paragraph{} 
On pense à ce stade que notre proposition assiste l'architecte dans le
developpement évolutionnaire du SdS. 
Cependant des points de l'expérimentation sont à améliorer. Notamment
le choix des architectes, dans cette expérimentation, nous n'avons
pas eu d'autres sujet que nous même. 
Et ce stade, les scénarios d'execution du SdS sont limités et
modélise des cas d'execution simple. 
 
