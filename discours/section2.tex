\section{Patron et processus de reconfiguration}

\subsection{Diapo :}

Pour la reconfiguration, on s'est fixé comme objectif l'application
de changement sur le sds pendant qu'il s'execute. 
D'après l'exemple, que l'on a developpé on va  s'intéresser à
maîtriser la qualité de service  et
s'assurer de la cohérence de certaine transaction pendant la
reconfiguration.

On a pris comme artefact de base pour raisonner sur les reconfigurations
d'un système, ça description en terme de composant et connecteur.
L'intérêt pour la reconfiguration et de pouvoir raisonner sur les
parties modulables du SdS.  cela permet d'identifier les parties
sujettes au changement et de voir leurs dependances dans le SdS.

On remarque que les opérations qui composent les scripts de
reconfiguration peuvent être décomposées en opération primitive et
organisées suivant l'importance des modications réalisées sur le SdS. 
Elles sont introspectives quand elle l'opération sert à oberser l'état
du SdS ou intercessives quant elles réalisent des modifications sur le
système. Les opérations sont : 
- ... 
- ...

La conception d'une reconfiguration consiste à produire un script
composé de ces opérations primitives. A travers la mise en \oe{}uvre
de ces opérations l'architecte cristalisent un certain nombre de
décision de conception qui ont un impacte sur la qualité des services
pendant la reconfiguration et la préservation de la cohérence des
transactions. 

Si on regarde un cas typique de stratégie de
reconfiguration qui est la quiescence. L'architecte veut mettre à
jours un composant sans interrompre les parties du système qui ne sont
pas dependantes de ce composant. L'architecte va décider de déconnecter
progressivement les composants qui dépendent du composant à
remplacer. Puis une fois que ts les composants dépendants sont
déconnectés, le composant ciblé par la reconfiguration peut être
remplacé sans rédémarrer complétement le système n'y perdre les
transactions démarrées.


Ce qui nous intéresse est de documenter ce type de stratégie de
reconfiguration pour la rendre réutilisable. Sur la base du type de
contenu des patrons de conception en général, un patron réutilisable
est décrit par son intention, contexte, problème, force, solution et
conséquence. 

Ce qui peut caractériser un problème de reconfiguration pour les sdss
est l'hétérogénéité des mécanismes de reconfiguration disponibles au
moment de la reconfiguration. Ce qui diverge des systèmes distribuées
puisque souvent les composants implémentent les mêmes mécanismes de
reconfiguration. Cela introduit de la variabilité également dans les
décisions que peut prendre l'architecte pour réaliser une même
stratégie de reconfiguration. Si on reprend l'exemple de la
quiescence, l'architecte peut décider une approche où le composant
ciblé par la reconfiguration atteind son état de quiescence
naturellement ou alors il peut décider suivant les mécanismes
disponibles de forcer l'état de quiscence pour terminer la
reconfiguration. 


Dans la suite, on va voir d'après l'état de
l'art comment l'architecte peut documenter sa reconfiguration de façon
réutilisable mais aussi quel processus d'ingénierie il a à sa
disposition pour assister le travail de conception d'une
reconfiguration quand les mécanismes de reconfiguration ne sont pas
anticipables.   

\subsection{Diapo : }
On peut classifier le type de documentation suivant son niveau de
formalité. On a des documentations qui sont très formelles comme
celles utilisées par Allen ou Oliveira pour documenter des
reconfigurations. Ce niveau de formalisme permet de vérifier
formellement qu'une propriété est vérifiée pendant la reconfiguration. 
Si on regarde les aspects réutilisations, on comprend l'intention de
la reconfiguration en analysant l'architecture ciblée, le contexte est
difficilement compréhensible si l'architecte ne connait pas le
formalisme, la solution est documentée mais difficile à interpreter si
on ne connait pas le formalisme, les forces et conséquences ne sont pas
clairement documentées. La réutilisation dans ce cas est partielle. 

Si on regarde des approches moins formelles. Les travaux de Gomaa sont
plus intéressants. L'intention de la reconfiguration est implicite ici
car les patrons ont tous pour objectifs de remplacer un composant avec
une stratégie de quiscence. Le problème est également implicite, il
s'agit préserver les ... (quiescence), par rapport au documentation
précedente, on voit que le contexte est documenté par des digrammes de
collaboration qui montrent le rôle des composants dans l'architecture. 
La solution est documentée via des diagrammes de collaboration qui
renseignent sur le rôle des composants entre eux. Par rapport à
l'approche précédente, la compréhension est simplifiée et elle ne
nécessite pas de connaissance pointue. 

\subsection{Diapo : }

A ce stade on voit que les approches comme celle de Gomaa sont du
point de vue de la réutilisation, une type de documentation plus
pertinent. 

On par rapport au critère de reconfiguration SdS, on voit que voit que
la solution de Gomaa est limité dans un contexte sds. en effet, on ne
peut pas faire l'hypothèse que les composants implémentent strictement
les mêmes mécanismes de reconfiguration. On veut proposer une
documentation qui prend en compte plus de type de décision de
reconfiguration. On voit que l'on aura besoin d'expliciter d'autres
champs pour notre concept de patron de reconfiguration.

Ensuite pour continuer notre approche de reconfiguration, on veut
s'intéresser au processus de reconfiguration que l'architecte a à sa
disposition dans l'état de l'art. 

\subsection{Diapo : à déplacer }

Dans un premier temps, cela consiste à considérer la reconfiguration
comme un aspect à part dans le developpement des systèmes. 
Donc considérer des langages et outil propre au
problème de reconfiguration. Ces langages assistent l'architecte pour
maintenir des propriétés pendant la reconfiguration. 
On voit que ces méthodes n'aide pas vraiment l'architecte à trouver
une solution mais plutôt à vérifier le script conçu. 

Des approches qui automatisent la tâche de reconfiguration est de
dériver automatiquement des scripts de reconfiguration à partir de la
descrition de l'architecture cible et source ne permettent pas 
maintenir pendant une reconfiguration. Dans ce cas, les solutions de
reconfiguration à traiter sont 
 
