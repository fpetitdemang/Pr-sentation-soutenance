\section{Patron et processus de reconfiguration}

\subsection{Diapo :}

\textit{Après étude de la réutilisation des documentations, on a définit comme
critère à la réutilisation  ... }



Sur la base du type de
contenu des patrons de conception en général, un patron réutilisable
est décrit par son intention, contexte, problème, force, solution et
conséquence. 



Dans la suite, on va voir d'après l'état de
l'art comment l'architecte peut documenter sa reconfiguration de façon
réutilisable mais aussi quel processus d'ingénierie il a à sa
disposition pour assister le travail de conception d'une
reconfiguration quand les mécanismes de reconfiguration ne sont pas
anticipables.   

\subsection{Diapo : }
On peut classifier le type de documentation suivant son niveau de
formalité. On a des documentations qui sont très formelles comme
celles utilisées par Allen ou Oliveira pour documenter des
reconfigurations. Ce niveau de formalisme permet de vérifier
formellement qu'une propriété est vérifiée pendant la reconfiguration. 
Si on regarde les aspects réutilisations, on comprend l'intention de
la reconfiguration en analysant l'architecture ciblée, le contexte est
difficilement compréhensible si l'architecte ne connait pas le
formalisme, la solution est documentée mais difficile à interpreter si
on ne connait pas le formalisme, les forces et conséquences ne sont pas
clairement documentées. La réutilisation dans ce cas est partielle. 

Si on regarde des approches moins formelles. Les travaux de Gomaa sont
plus intéressants. L'intention de la reconfiguration est implicite ici
car les patrons ont tous pour objectifs de remplacer un composant avec
une stratégie de quiscence. Le problème est également implicite, il
s'agit préserver les ... (quiescence), par rapport au documentation
précedente, on voit que le contexte est documenté par des digrammes de
collaboration qui montrent le rôle des composants dans l'architecture. 
La solution est documentée via des diagrammes de collaboration qui
renseignent sur le rôle des composants entre eux. Par rapport à
l'approche précédente, la compréhension est simplifiée et elle ne
nécessite pas de connaissance pointue. 

\subsection{Diapo : }

A ce stade on voit que les approches comme celle de Gomaa sont du
point de vue de la réutilisation, une type de documentation plus
pertinent. 

On par rapport au critère de reconfiguration SdS, on voit que voit que
la solution de Gomaa est limité dans un contexte sds. en effet, on ne
peut pas faire l'hypothèse que les composants implémentent strictement
les mêmes mécanismes de reconfiguration. On veut proposer une
documentation qui prend en compte plus de type de décision de
reconfiguration. On voit que l'on aura besoin d'expliciter d'autres
champs pour notre concept de patron de reconfiguration.

Ensuite pour continuer notre approche de reconfiguration, on veut
s'intéresser au processus de reconfiguration que l'architecte a à sa
disposition dans l'état de l'art. 

\subsection{Diapo : à déplacer }

Dans un premier temps, cela consiste à considérer la reconfiguration
comme un aspect à part dans le developpement des systèmes. 
Donc considérer des langages et outil propre au
problème de reconfiguration. Ces langages assistent l'architecte pour
maintenir des propriétés pendant la reconfiguration. 
On voit que ces méthodes n'aide pas vraiment l'architecte à trouver
une solution mais plutôt à vérifier le script conçu. 

Des approches qui automatisent la tâche de reconfiguration est de
dériver automatiquement des scripts de reconfiguration à partir de la
descrition de l'architecture cible et source ne permettent pas 
maintenir pendant une reconfiguration. Dans ce cas, les solutions de
reconfiguration à traiter sont 

\subsection{Diapo : Patron de reconfiguration - délimitation du problème}

Le problème de reconfiguration se réduise à la description du passage
d'une architecture à une autre. 
On peut cependant, raffiner en différent problème suivant les
solutions dévéloppés et en abstraire les régularités pour définir des
patrons de reconfiguration. 

On va essayer de déduire l'ensemble des besoins que cherche à
satisfaire la solution pour capturer le problème exprimer par le
patron. 
On a trouvé qu'un bon moyen est d'adord d'expimer l'architecture
intiale, puis celle ciblée. 
L'architecture est bon support pour la compréhension du problème lié à
qualité de service puisqu'elle modélise les dépendances fonctionnelles
enntre les composants. 

Ensuite on propose de documenter les contraintes que doivent satisfaire
la solution. Elles peuvent être structurelles ou comportementales.

Les forces doivent expliquer les principes mis \oe{}vre dans la
solution. Cela doit aider l'archticte à comprendre quelles sont les
points importants de la solution.  

\subsection{Diapo : Patron de reconfiguration - rédaction de la solution}

Ensuite la solution consiste à décrire les mécanismes d'évolution
et d'une grammaire de reconfiguration. La grammaire de reconfiguration
consiste à  décrire les opérations de reconfiguration et leur ordre
afin de d'execution. Pour documenter la grammaire de reconfiguration
nous nous sommes appuyé sur des diagrammes de collaboration et d'état
ainsi qu'une description textuelle. 


\subsection{Diapo : Patron de reconfiguration - amélioration de la
ré-utilisation}

Naturellement chaque solution à un impacte sur les aspects non
fonctionnels du système pendant la reconfiguration qui doivent être
documenté. 

Pour finir, la documentation doit permettre à l'architecte d'évaluer
rapidement la pertinence de la solution sans avoir à comprendre en
détail le patron de reconfiguration. Dans l'idée de titre des patrons, on a garder l'idée de la métaphore.
Elle intéressante pour aider l'architecte dans la compréhension du
patron de reconfiguration. On a par exemple,
\begin{itemize}
\item quiescence 
\item co-evolution
\item re-routage
\item tranquilité
\end{itemize}

Dans la continuité du titre l'intention doit aboutir à la
comprenhension de la métaphore. en expliquant, un approche naive et
ses inconvénients puis on explique l'idée générale du patron et ses
avantages en comparaison avec l'approche naive. 
l'approche du patron  

\subsection{résumé } 
