\section{Processus de reconfiguration}

\subsection{Processus de reconfiguration existant}
\paragraph{}
Une fois que l'on a documenté des solutions de reconfiguration
réutilables on s'est interessé au processus de conception.

\paragraph{}
Une approche dans la conception des reconfigurations est automatique.
L'architecte fournit l'architecture ciblée et la reconfiguration est
dérivée automatiquement. C'est le cas de la
proposition de Boyer. On voit que l'algorithme de reconfiguration se
compose de deux phases, la prémière phase réalise ttes les opérations
de déconnexion, d'arêt et de suppression de composant, puis la seconde phase
réalise toutes les opérations de connexion, création et démarrage.
Ensuite si on détaille les opérations d'arêt, son arêt impliquent un
comportement récursif qui propagent l'opération d'arêt à tous les
composants qui dépendent du composant ciblé par l'opération d'arêt. 

\paragraph{}
Dans d'autres cas, l'approche est manuelle, dans la contribution de
buisson, l'architecte doit fournir
une reconfiguration et une preuve vérifier pour apppliquer la reconfiguration. 
Comme expliqué par l'auteur, le processus de reconfiguration est
itératif, car en pratique l'architecte fait plusieurs itérations, soit
pour refaire la reconfiguration ou soit pour refaire la preuve de sa
reconfiguration.

\paragraph{}
De façon général, on voit dans l'état de l'art que les processus de
reconfiguration sont plus ou moins artisanals. Comme dans un contexte SdS,
l'architecte est obligé d'adapter ses hypothèses de reconfiguration à
cause de l'indépendance des CSs. Dans ce sens, pour aider l'architecte
à réaliser des reconfigurations on a proposé un processus où l'on a
identifier les étapes de la conception d'une reconfiguration. 


\subsection{Proposition d'un processus de reconfiguration}

\paragraph{}
A partir de l'étude des pratiques de reconfiguration, on a proposé un
processus de reconfiguration décomposé en trois phases : 
\begin{itemize}
\item préparation : opérations de reconfiguration préalable au
opération. Il s'agit d'ajouter des mécanismes d'évolution comme
l'ajout de tampons par exemple ou mécanisme de synchronisation. 
\item modification : action souhaité intialement. les oéprations de
reconfiguration qui vont permettre les modifications souhaités par
l'architecte. 
\item nettoyage : suppression des mécanismes d'évolution liés à la
reconfiguration qui ont été déployés dans la phase de prépration.  
\end{itemize}

\paragraph{}
Ensuite on a considéré que chaque phase de prepration peut être également
décomposé en phase : preparation, modification et nettoyage. 


\subsection{Utilisation des patrons de reconfigiration}

\paragraph{}
Ici l'intérêt d'expliciter la récursion dans le processus est de
permettent à l'architecte de contrôler la qualité de service
explicitement pendant la reconfiguration en décidant de relâcher des
contraintes pour permettre la reconfiguration. 
%
C'est dans le cas, où les contraintes de reconfiguration rendent
impossible la reconfiguration. Dans ce cas l'architecte peut décider
d'une nouvelle itération dont l'architecture ciblé et l'architecture
intégrant les contraintes suffisantes pour appliquer les
reconfiguration décidé dans l'itération précédente. 

\paragraph{}
Le processus s'appuie sur les patrons de reconfiguration. Dans un
premier temps, l'architecte analyse les modifications à apportées. 

il utilise le catalogue de patron pour identifier les problèmes qu'il
peut rencontrer et comment les résoudre. ou il définit un nouveau
patron. 

Dans le cas, ou le solution n'est pas directement applicable alors il
peut décider d'une nouvelle itération. Dons la cible est l'archicture
requise pour appliquer la reconfiguration de l'itération précédente. 



