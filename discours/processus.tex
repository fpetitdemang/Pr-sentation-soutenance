\section{Processus de reconfiguration}

\subsection{Processus de reconfiguration existant}

Une fois que l'on a documenté des solutions de reconfiguration
réutilables on s'est interessé au processus de conception.

Une approche dans la conception des reconfigurations est de
dériver de fournir en entrée du processus une architecture initiale et
une architecture ciblée. Dans ce cas, il existe des approche
automatique où la reconfiguration est dérivée automatiquement à partir de la
descrition de l'architecture cible et source. C'est le cas de la
proposition de Boyer, dans son papier, elle propose à partir de la
différence architecturale de l'archi. source et cible de dérivé toutes
les opérations de reconfiguruation (les connexions ajouter et
supprimées) puis défini suivant : déconnexions des connecteurs
optionnels, arêt composant, déconnexion des connexions mandataires.
puis : supprssion des composants, création des nouveaux, connexions,
démarrée.   

Dans d'autres cas, elle
définie manuellement mais aucune approche n'est fournie pour assister
l'architecte dans la conception. L'assistance est fournie par des
outils basés sur des méthodes formelles (model checking, preuve) pour
vérifier les propriétés de la reconfiguration. Il ny'a pas d'aide
particulière pour assister l'architecte pour la phase de conception. 

Dans l'état de l'art de la conception des reconfigurations, on
remarque les reconfigurations interviennent dans des environnements de
reconfiguration connus à l'avance. Dans un environnement SdS,  
Cela facilite l'automatisation de la conception des reconfigurations,
l'architecte à besoin de fournir in fine l'architecture ciblée par la
reconfiguration.


\subsection{Processus de reconfiguration pour les SdSs}

%supporte des opérations de reconfiguration hétérogène
Dans un contexte SdS, les mécanismes de reconfiguration ne sont pas
homogènes et sont succeptibles d'évoluer ainsi que les exigences de
reconfiguration. Par conséquent, l'architecte doit reprendre la
conception du script, sachant que les approches automatiques font
l'hyothèse que ttes les opérations de reconfiguration sont
disponibles. 

%supporte les contraintes dynamiques.
Par conséquent, la conception d'une reconfiguration dans un contexte
SdS devient plus complexe. On propose un processus de reconfiguration
qui permette à l'architecte d'expliciter les dégradations de service
qui peuvent survenir pendant la reconfiguration. On va considerer que
les dégradations de services peuvent momentanné impacter des parties
de l'architecture différente dans le temps. 



\subsection{Proposition d'un processus de reconfiguration}
Dans ce contexte, on a proposé un processus de reconfiguration qui
tiennent compte :
- des 

On a proposé dans la thèse d'expliter un processus qui prennent en
compte ces contraintes dynamiques et ainsi de considérer le dégradation
de service pendant la reconfiguration. mais aussi la variabilité des
décisions qui peuvent intervenir à cause de l'environneent de
reconfiguration hétégrogène. 

On a proposé un processus itératif et récursive. Donc l'idée est que
la reconfiguration se compose d'au moins trois phases : 
preparation : opérations de reconfiguration préalable au opération ...
modification : action souhaité intialement
nettoyage : suppression des mécanismes liée à la reconfiguration 

On a considéré que chaque phase de prepration peut être également
décomposé en phase : preparation, modification et nettoyage. 


\subsection{Utilisation des patrons de reconfigiration}

Le processus s'appuie sur les patrons de reconfiguration. Dans un
premier temps, l'architecte analyse les modifications à apportées. 

il utilise le catalogue de patron pour identifier les problèmes qu'il
peut rencontrer et comment les résoudre. ou il définit un nouveau
patron. 

Dans le cas, ou le solution n'est pas directement applicable alors il
peut décider d'une nouvelle itération .
