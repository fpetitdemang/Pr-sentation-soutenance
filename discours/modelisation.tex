\section{Framework pour la modélisation des SdSs}

\subsection{Diapo 12}
\paragraph{à déplacer}
On a vu qu'une problématique de cette thèse a été la modélisation
d'architecture de SdSs nécessaire à l'étude de la reconfiguration
dynamique.
%
Le besoin ... 
%
Les modèles d'une manière générale sont des abstractions de la
réalité. 
%
En génie logiciel, il capture l'architecture du SdS. 
Ils sont utilisés pour améliorer la qualité et le cout des systèmes. On
peut réduire les cout et augmenter qualité par exemple avec
l'utilsiation des modèles pour générer automatiquement l'architecture
d'un système, faire des simulations pour améliorer la qualité de
l'architecture avant de la déploiyer. 

\paragraph{}
Un objectif de modélisation des SdSs est d'obtenir des modèles
exactes et précis. 
%exactitude
Ce qui nous interesse est de réussir à modéliser les abstractions qui
concour à la réalisation du SdS. On s'est interessé en particulier à
identifier tous les CSs et les ressources qu'ils déploient ainsi que
leurs objectifs d'interaction.
%précision
On aura besoin de modèle suffisament précis pour modéliser les
décisions
architecturales du SdS.

\paragraph{}
La problématique est un contexte de modélisation difficile dû  aux frontières du SdS flou.
Les frontères sont flou car les composants du SdS sont développés par
des tiers. Si on regarde la figure, si le SdS définit ses frontières
aux composants A,B, et C, il ne prend pas en compte D qui si il se
désengage de A et C peut dégrader fortement le fonctionnement global
du SdS.  

\paragraph{}
Parmis les études de cas disponibles peu s'intèressent à la
reconfiguration dynamique mais plutôt à la conception des
architectures. On s'est plutôt intéressé au langage et processus de
modélisation mis en \oe{}vre dans les études de cas. 
%
Pour la conception des systèmes complexes comme les sdss les langages
de modélisation ont été développé pour gérer la compléxité des
modèles, la précision et l'exactitude. Dans cette section, nous allons
étudier les langages de modélisation dans le developpement des SdSs et
leur portée. On définira leurs caractéristiques utiles et comment
elles sont exploiter dans les conceptions des SdS à travers deux
projets Europééens qui sont DANSE et COMPASS.


\subsection{Diapo 13}
\paragraph{}
Un premier langage de modélisation auquel on s'est intéressé est
SySML. C'est un langage de modélisation qui a l'ambition de permettre
une approche de modélisation holistique. L'idée est que le langage
puisse supporter toute les phases de développpement du SdS, notamment
en intégrant tous les corps de métier qui peuvent intervenir dans le
développement d'un système. 

\paragraph{}
Les principaux éléments de modélisation utilisés sont la notion de
block et de port. 
Les blocks modélisent des unités modulaires d'un système. 
Les ports décrivent les points d'intercation entre les blocs. Il y
plusieurs catégorie de port qui peuvent permettre de modéliser des
interactions entre des parties logicielles et matérielles. 
Ensuite les flux ajouter une description physique des interactions
entre les blocs. 
Pour finir la relation d'allocation permet de modéliser des relations
de raffinement entre les éléments modéliser ce qui est utile pour
supporter les processus de modélisation descendant ou ascendant. 

\paragraph{à déplacer} Plusieurs type de diagramme interviennent : les
diagrammes de block, d'interaction, d'activité, paramétrique,
d'exgience. 

\paragraph{} Les avantages pour la modélisation des systèmes de
systèmes sont la description des connectivités puisque la modélisation
des SdSs s'interessent à la description des interactions entre les
CSs, et traçabilité entre les éléments de modélisation qui permet
de définr des relations entre des modèles developpé à différent niveau
d'abstraction et/ou différent aspect, ce qui est un avantage des SdSs
qui intègre différent parti prennant dans la modélisation des SdSs. 

\paragraph{} Evidemment l'inconvénient est l'absence de parti pris
sur le type des éléments à modéliser. Des frameworks de modélisation
Comme UPDM peuvent vu comme des profils de SySML est proposer des
modèles pour le developpement des SdSs. 

\subsection{Diapo 14} 
\paragraph{}
un second langage que l'on a étudié est UPDM. UPDM est profil de
SysML. Sa principale contribution est la proposition de 40 vue pour la
modiélisation de SdS. 

\paragraph{}
Différent type de vue propose de modéliser des niveaux d'abstraction
différent :
\begin{itemize}
\item Les vues Acquisition / Project (AcV / PV) décrivent les aspects
organisationnels
des projets, y compris leur calendrier et les jalons.
\item Les All-Views (AV) rassemblent des informations globales et des
métadonnées sur
les éléments de l’architecture.
\item Les vues opérationnelles (OV) sont une collection de vues qui
décrivent les activités
impliquées dans le SdS, ainsi que les ressources (humaines ou
mécaniques) qui
exécutent ces activités.
\item Les vues orientées services (SOV) décrivent les services, en termes
d’interfaces,
ainsi que les fonctions que les services sont censés exécuter dans le
but de mettre
en œuvre les activités décrites dans les vues opérationnelles.
\item Les vues stratégiques / de capacités (StV / CV) sont un ensemble de
vues à
long terme croisées entre projets qui décrivent et organisent les
capacités d’une
organisation, en prévision de projets.
\item Les vues système / services (SV / SvcV) sont une collection de vues
qui décrivent
comment les capacités opérationnelles, décrites dans les vues
opérationnelles, et
les exigences des utilisateurs peuvent être réalisées en termes de
capacités d’équi-
pement (ou humain), c’est-à-dire la spécification des constituants
impliqués dans
le système.
\item Les vues techniques / standards (TV / StdV) décrivent les
technologies, les règles
et les normes qui sous-tendent la mise en œuvre du système,
fournissant ainsi un
outil pour anticiper les progrès technologiques ou les perturbations
qui peuvent
affecter le système. 
\end{itemize}

\paragraph{}
Concretement on a regardé leur utilisation pour définir leur portée et
comment ils sont utilisés.  Pour cela on a étudié les principaux
projets de recherche européen
dans le domaine de SdSs. Il s'agit de DANSE. qui a basé sont projet
developpemment de SdS sur le langage UPDM. 

\paragraph{} 
Le tableau fait la synthèse des vues et diagrammes selectionnées par
danse.
 
Si on regarde leur utilisation, les vues opérationnelles servent à
identifier .... 

les vues systèmes servent à ...

\paragraph{}
après la phase de modélisation avec UPDM, les modèles sont raffinés
vers des représentation plus formelles. définition des contrats sur
l'architecture comme objectif à optimiser pour générer une architecture, 
la vue sv-1 assiste la génération d'architecture automatique.
définition de contrainte de cout à respecter, ...

les vues sv-4 servent à analyse les evolutions de possible du SdS.
définition de relation de causalité et probabilité de réalisation. 
+ Définition de contrat sur les CSs 

\paragraph{}
Si on regarde en comparaison le framework developpé par compass, les
vues developpées sont similaire.
 
\paragraph{}
A ce stade on voit que l'exactitude est obtenu grâce à un processus de
développement descendant. La modélisation démarre par des abstractions
hauts niveaux qui capture les systèmes participants et les objectifs
d'interaction. Puis raffine sont raffiné vers des abstractions
concrètrs. 
Au niveau de la précision on a noté que SySML et UPDM ne sont pas
suffisant pour exprimer des propriétés vérifibles de façon automatique. 


\subsection{}

\paragraph{}
Dans notre approche de modélisation des architectures de SdSs on a
choisi de se baser sur l'existant. Sans proposer de modification des
profondes. En effet, en comparaison avec d'autres travaux de
modélisation que soit pas basé sur UPDM on voit que les vues et pts de
vue sont similaires. On a pris en parti pri que cela suffit pour le
moment à fournir des données réalistes. 


\paragraph{} 
On propose donc de suivre une approche top down pour la modélisation.
- On a ajouté une phase de modélisation des exigences pour les
structurer. 
- La phase de modélisation du niveau SdS repose sur les vues ov1, ov1b,
ov5.
- On a explicité une phase de validation manuel. 
- Puis une phase de modélisation de l'architecture au niveau CS.
- Puis une phase de vérification de la cohérence de l'architecture
avec une simulation. 

\paragraph{}
Si on regarde la modélisation du SdS on a retenu les vues OV-1, OV-1b,
OV4 et OV-5. 
%
On a rejoint le parti de DANSE, cependant on a trouvé plus
pertinent l'utilisation du diagramme de cas d'utilisation pour
modéliser les objectifs d'interaction entre les resssources. Cela
améliorer la lisibilité du diagramme et permet de guider le
developpement de la description des fonctions du SdS. Le détaille de
la connectivité modélisé avec OV2 dans DANSE description de la 
connectivité est décrite dans la vue OV-4.

\paragraph{} 
La phase de modélisation du niveau CS ...
ajout de contrainte ocl pour préserver l'expressivité du langage on a
opté pour l'utilisation de primitive architectural spécifié en ocl ce
qui nous a permis de l'intégrer directement dans la vue sv-1, utilisation de la sémantique du pi-calcul.

\paragraph{} 
A ce stade on a clarifié les choix de modélisation et les raisons.
Maintenant je vais expliquer configurations que l'on a choisi de
modéliser. 
\subsection{Diapo 17}
 
\paragraph{} 
On a choisi comme cas d'étude le système de service de secours
d'urgence. Le domaine du système est un cas typique de SdS. 

\paragraph{}
La mission principale sur laquelle se concentre notre cas d'étude est la
protection des personnes, des biens et de
l’environnement. Il peut s'agir du
transport d'une personne à l'hôpital, de limiter la progression d'une
inondation, d'un incendie ou de circonscrire une pollution fluviale. De manière
générale, un des objectifs de la mission est d'empêcher l'évolution d'un sinistre et de revenir à
une situation normale.   

\paragraph{} 
Les principaux cs sont : 
\begin{itemize}
%\item L'organe de contrôle principal est le Centre Opérationnel Départemental d'Incendie et de Secours (CODIS) qui traite les appels des victimes, supervise et coordonne l'ensemble des activités opérationnelles d'un service départemental d'incendie. Dans notre exemple deux centres d'opérations sont impliqués CODIS56 et CODIS35
\item Les hôpitaux prennent médicalement en charge  les victimes.
\item Les pompiers, à l'échelle départementale, sont structurés par un Service
Dépar\-te\-mental d'Incendie et de Secours (SDIS) qui possède ses
propres ressources maté\-rielles et sa propre structure de commandement.
Ce service est  responsable du transport des blessés, de la lutte contre les
incendies et de la protection des biens.  
\item Le  Service d'Aide Médicale
Urgente (SAMU) est le centre de régulation médicale d'urgence qui régule les
ressources de soins d'urgence (ambulances par exemple).
\item Enfin, la sécurité civile est impliquée dans des situations de crises majeures et peut fournir des moyens aériens.
\end{itemize}



\paragraph{} 
Au niveau du territoire français, les services de secours sont organisés par
département. Chaque type de service est déployé sur chaque département français
avec son propre financement et sa propre structure hiérarchique de commandement. La collaboration entre
les services est favorisée par le réseau
Il s'agit du
 réseau numérique de communications intra-services
(quand les communications sont limitées aux membres d'un même service) et inter-services (quand
les communications permettent à des membres de services différents de communiquer). Les
collaborations que peuvent former le sos ont des échelles différentes.
Elles peuvent être~: 
\begin{itemize}
\item Locales quand il s'agit de
communications entre deux ressources proches géographi\-quement. 
\item Départementales quand les ressources qui communiquent sont distribuées à l'échelle du
département. 
\item Inter-départementales quand ce sont des ressources de services appartenant
à deux départements différents.
\item Nationales quand elles impliquent des membres du gouvernement. 
\end{itemize} 

\paragraph{} 
Une gestion efficace consiste à
prendre les bonnes décisions au bon moment. Cette efficacité dépend de la
qualité des informations et du moment où elles sont fournies. Pour ces raisons
les services de secours se structurent autours d'une chaîne de
commandement. Les composants de la chaîne de commandement (qui sont les
ressources des services de secours) sont partionnés suivant leur niveau de
responsabilité décisionnaire. Ces niveaux de décision sont d'ordre : 
\begin{itemize}
\item Stratégique. Cela consiste à définir les objectifs et anticiper les besoins opéra\-tionnels en
réservant des ressources, afin de les allouer au moment opportun.
\item Tactique. Cela consiste à assigner des
objectifs aux ressources allouées
pour réaliser les objectifs stratégiques. 
\item Opérationnel. Cela consiste à décider d'opérations primitives dans le but
de réaliser les objectifs tactiques. 
\end{itemize}

La discipline hiérarchique impose aux composants de n'interagir qu'avec le niveau n+1 ou n-1. Lorsqu'il
s'agit d'interaction avec les composants du niveau n+1, ces interactions sont des remontées
d'information. Par exemple, les composants du niveau opérationnel font une
synthèse de la situation observée aux composants du niveau tactique. Les
communications vers le niveau n-1 concernent des directives. Par exemple, les
composants du niveau tactique décident de la zone de positionnement d'un
composant opérationnel et des actions qu'il doit réaliser. 
%
\paragraph{}
Cette chaîne de commandement se reflète dans la structure du réseau de
communication des pompiers. La communication entre le CODIS et ses
ressources déployées se fait principalement via
un canal de communication radio appelé canal opérationnel par le réseau ANTARES. Le canal opéra\-tionnel est un groupe de
communication déployé pour chaque SDIS. Chaque membre du groupe connecté
sur le même canal de communication peut écouter les messages envoyés par les autres
membres du groupe. On parle d'écoute passive. Via le canal opérationnel, le CODIS gère
toutes les interventions courantes en cours dans le département :
\begin{itemize}    
\item Les communications se font seulement de façon verticale. Par
exemple, un VTU ne peut pas communiquer avec un autre vtu.
\item Les abonnés ne prennent la parole que s'ils ont l'autorisation du CODIS et
s'il n'y a pas une autre communication en cours. Par exemple, le VTU demande
l'autorisation de parler avant d'envoyer son rapport. Un seul abonné peut avoir
la parole à un moment donné. 
\end{itemize}

D'autres types de canaux sont disponibles comme le canal tactique et
stratégique, pour les niveaux correspondants de la chaîne de commandement. Ces
canaux fonctionnent de la même manière que les canaux opérationnels, avec les mêmes contraintes.

\paragraph{}
La première configuration considère que deux sdis ont une collaboration
opérationnelle inter\-dé\-par\-tementale,
c'est-à-dire qu'ils communiquent directement, seulement pour ce qui concerne les
actions de terrain réalisées pour remplir la mission. Une telle collaboration
est mise en place, par exemple, lorsqu'un SDIS fait face à un manque de
personnel, et demande donc des ressources manquantes auprès d'un SDIS voisin
pour une mission de petite taille. Dans notre scénario, cette configuration est
utilisée lorsque les habitants appellent des services d'urgence en réponse à ce
que l'on pense être une inondation localisée.

\paragraph{}
Dans la deuxième configuration, deux SDIS ont une collaboration tactique et
stratégique interdepartementale, c'est-à-dire qu'une chaîne de commandement hiérarchique est mise en
place. En plus de la collaboration opérationnelle, la collaboration tactique
permet aux commandants de communiquer avec leurs subordonnés, afin de leur
donner des instructions et de recueillir des rapports sur les actions sur le
terrain. La collaboration stratégique garantit que le SDIS collaborateur adopte
des orientations cohérentes afin d'atteindre les objectifs de niveau supérieur
définis par la mission. Cette deuxième configuration est utilisée lorsque la
crise est plus importante, ce qui nécessite plus de ressources que dans la
première configuration, et lorsque ces ressources nécessitent une coordination
étroite. Dans notre scénario, le SdS évolue vers cette seconde configuration
lorsque les services d'urgence découvrent que l'inondation est plus critique
qu'ils ne le pensaient initialement.

\paragraph{}
Enfin, dans la troisième configuration, un SDIS collabore avec le SAMU et la
sécurité civile. Cette configuration se produit lorsque les services d'incendie
et de secours ont besoin de l'aide d'autres services pour faire face à la crise.
Dans notre scénario, le SAMU est impliqué afin de réguler l'évacuation des
blessés vers les hôpitaux les plus proches, et la sécurité civile fournit des
ressources de type hélicoptère lorsque l'inondation rend les routes
inutilisables.

\paragraph{} 
Maintenant que l'on a identifié des candidats à la modélisation, on
peut appliquer notre framework de modélisation. 


\subsection{Diapo }

\paragraph{}
étape 1 

\paragraph{}
étape 2 

\paragraph{}
étape 3 

\paragraph{}
étape 4 

\paragraph{}
étape 5 

\subsection{Diapo }

\paragraph{} 
A ce stade, nous avons modélisé 3 architectures qui correspondent à
la description d'environ ... composant/connecteur. et xx diagrammes.

Le processus de modélisation est supporté par le langage UPDM. On a vu
que UPDM n'est pas utilisable tel quel. il a fallu selectionné les
vues les plus pertinantes parmi la quarantaine disponibles. Pour
confirmer nos choix nous nous somme basé sur le choix de modélisation
d'autre étude de cas. 
%
Cela nous a permis d'expliciter des phases de
modélisation utilise. 
%
La modélisation des architectures reposent sur
l'utilisation de primitive architecturale ocl. L'avantage est
d'ajouter la précision nécessaire à une vérification automatique des
décisions architecturales. mais aussi d'assister la vérification de
l'exactitude de l'architecture en rendant plus expressif les
contraintes ocl. 

Cependant on peut noter tout de même que les modèles ne sont pas
complets  et pourrait être plus détaillés. 
