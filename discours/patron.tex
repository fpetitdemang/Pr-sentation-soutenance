\section{Patron et processus de reconfiguration}

\subsection{Diapo : Réutilisation de décision de reconfiguration}

Pour définir des critères de documentation réutilisables, on a fait la
synhtèse des parties descriptives des patrons de conception dans
l'état de l'art. On en en tirer la liste suivante. 
\begin{itemize}
\item titre
\item intention,
\item contexte, 
\item problème, 
\item solution 
\item conséquence.
\end{itemize} 

\paragraph{} 
Le titre et l'intention on pour objectif de donner l'intuiton du
principe mis en \oe{}uvre par la solution documentée. En général, le titre
correpond à une métaphore du principe suivi par la solution et
l'intention explique pourquoi une solution naive ne
fonctionne pas et quel est un principe plus adapté.   

\paragraph{} 
Le contexte va décrire quelles sont les hypothèses considérées par le
patron.  En somme, les pré-requis et les facteurs qui empêcheraient
l'application de la solution. 

\paragraph{} 
Le problème va expliquer quels sont les besoins que cherchent à
satisfaire la solution. et quelles sont les forces mise en \oe{}vre
par la solution. 

\paragraph{} 
Ensuite la solution documente comment est résolue le problème et enfin
les conséquences expliquent en générale quels sont les effets sur les
aspects non fonctionnels ou les considérations d'implémentation. 


\subsection{Diapo : Documentation formelle Allen}
On peut classifier le type de documentation suivant son niveau de
formalité. On a des documentations qui sont très formelles comme
celles utilisées par Allen ou Oliveira pour documenter des
reconfigurations. Ce niveau de formalisme permet de vérifier

%%%%
Si on regarde la proposition de Allen, il propose  de
réutilisation en considérant la phase de reconfiguration comme un
aspect à part dans le processus de developpement d'un système et donc
en proposant un langage et des outils dédiés à la reconfiguration
dynamique. 
%

\paragraph{} 
Il propose pas de patron en particulier. Si on regarde comment il
spécifie une reconfiguration. on voit que : 
%
\paragraph{} 
le contexte est définit par le langage WrightADL qui base sa
sémantique sur CSP. 
Le contexte se caractérise par une description composant/connecteur.
Ici on voit que contexte est décrit par une architecture
client/serveur.
%
\paragraph{} 
le problème est spécifié par des contraintes structurelles à respecter. pendant la
reconfiguration. On voit qu'un composant serveur doit toujours être connecté à
un composant client.  
%
\paragraph{} 
la solution est documenté par les événements qui déclenchent les
opérations de reconfiguration. On voit que lorsqu'un événement capture
que le serveur principal est hors service, la reconfiguration consiste
à déconnecter le client du serveur principal, vers le serveur
secondaire.
%
\paragraph{} 
les conséquences discuttent des aspects techniquent : notamment de
l'implémentation des événements et de la distribution du connecteur.  

formellement qu'une propriété est vérifiée pendant la reconfiguration. 

\subsection{Diapo : Documentation formelle}
%%%%%
\paragraph{} 
Si on regarde en détaille la proposition de Oliveira. 
%
Si on regarde les aspects réutilisations, elle concerne la description
d'opération de reconfiguration primitive récrruente et la description
formelle qui l'accompagne.  
%
\paragraph{} 
Le titre décrit directement l'opération appliquée.
%
\paragraph{} 
Par rapport à nos critère la reconfiguration est trop bas niveau pour
décrire une intention de reconfiguration. 
%
\paragraph{} 
le contexte est définit par la description de primitive de
reconfiguration. qui sont : constante, parallèlisation, union,
découpage, suppression 
%
\paragraph{} 
Le problème, il est décrit par l'architecture source et ciblée. mais
il y pas vraiment de contrainte à respecter.... si ce n'est qu'à la
fin l'architecture ciblée est atteinte.  
%
\paragraph{} 
la solution est documentée mais difficile à interpreter si
on ne connait pas le formalisme, de plus elle ne résoud pas de
problème particilier. 
%
\paragraph{} 
les conséquences ne sont pas
clairement documentées. comme le niveau est bas elle ne capture pas de
conséquence.   
%
\paragraph{} 
La réutilisation dans ce cas est partielle. Dans ce, c'est
prinpalement car les patrons sont trop bas niveau pour être
réutilisés. ils ne capturent pas de problème de conception. 

%%% 
\paragraph{} 
Les documentation consiste à considérer la reconfiguration
comme un aspect à part dans le developpement des systèmes. 
Donc considérer des langages et outil propre au
problème de reconfiguration. Ces langages assistent l'architecte pour
maintenir des propriétés pendant la reconfiguration. 

\subsection{Diapo : Documentation semi-formelle}
\paragraph{}
Si on regarde des approches moins formelles. Les travaux de Gomaa sont
plus intéressants. Les patrons de reconfiguration définissent une
solution de reconfiguration par rapport à un problème de
reconfiguration qui est la quiescence. 

\paragraph{}
Le titre du patron fait référence à contexte de reconfiguration et au
principe mis en \oe{}uvre. 

\paragraph{}
L'intention et le problème de la reconfiguration sont implicites ici
car les patrons ont tous pour objectifs de remplacer un composant avec
le principe de quiscence.

\paragraph{}
par rapport au documentation
précedente, on voit que le contexte est documenté par des digrammes de
collaboration qui montrent le rôle des composants dans l'architecture. 

Ensuite la solution est documentée via des diagrammes de collaboration qui
renseignent sur le rôle des composants entre eux. 

\paragraph{}
Par rapport à
l'approche précédente, la compréhension est simplifiée et elle ne
nécessite pas de connaissance pointue. 

\subsection{Diapo : }

\paragraph{}
A ce stade on voit que les approches comme celle de Gomaa sont du
point de vue de la réutilisation, une type de documentation plus
pertinent. 

\paragraph{}
On par rapport au critère de reconfiguration SdS, on voit que voit que
la solution de Gomaa est limité dans un contexte sds. en effet, on ne
peut pas faire l'hypothèse que les composants implémentent strictement
les mêmes mécanismes de reconfiguration. On veut proposer une
documentation qui prend en compte plus de type de décision de
reconfiguration. On voit que l'on aura besoin d'expliciter d'autres
champs pour notre concept de patron de reconfiguration.

\paragraph{}
Ensuite pour continuer notre approche de reconfiguration, on veut
s'intéresser au processus de reconfiguration que l'architecte a à sa
disposition dans l'état de l'art. 


 
\subsection{Diapo : Patron de reconfiguration - délimitation du problème}

\paragraph{}
Le problème de reconfiguration se réduit à la description du passage
d'une architecture à une autre. 
On peut cependant, raffiner en différent problème suivant les
solutions dévéloppés et en abstraire les régularités pour définir des
patrons de reconfiguration. 

\paragraph{}
On va essayer de déduire l'ensemble des besoins que cherche à
satisfaire la solution pour capturer le problème exprimer par le
patron. 
On a trouvé qu'un bon moyen est d'adord d'expimer l'architecture
intiale, puis celle ciblée. 
L'architecture est bon support pour la compréhension du problème lié à
qualité de service puisqu'elle modélise les dépendances fonctionnelles
enntre les composants. 

\paragraph{}
Ensuite on propose de documenter les contraintes que doivent satisfaire
la solution. Elles peuvent être structurelles ou comportementales.

\paragraph{}
Les forces doivent expliquer les principes mis \oe{}vre dans la
solution. Cela doit aider l'archticte à comprendre quelles sont les
points importants de la solution.  

%%ù
\paragraph{}
Dans la littérature, on a essayé d'analyser des solutions
récurrente. On a constaté qu'une solution consistait à appliquer des
reconfiguration graduellement. On a définit comme contexte
architectural ..., puis comme contrainte que ..., et force ...  

\subsection{Diapo : Patron de reconfiguration - rédaction de la solution}

\paragraph{}
Ensuite la solution consiste à décrire les mécanismes d'évolution
et d'une grammaire de reconfiguration. La grammaire de reconfiguration
consiste à  décrire les opérations de reconfiguration et leur ordre
afin de d'execution. Pour documenter la grammaire de reconfiguration
nous nous sommes appuyé sur des diagrammes de collaboration et d'état
ainsi qu'une description textuelle. 
 
\paragraph{}
On a documenté la reconfiguration à partir du schéma ... 
et d'une description textuelle ... qui indique la sémantique des
opérations. 

\subsection{Diapo : Patron de reconfiguration - amélioration de la
ré-utilisation}

\paragraph{}
Naturellement chaque solution à un impacte sur les aspects non
fonctionnels du système pendant la reconfiguration qui doivent être
documenté. 

\paragraph{}
Pour finir, la documentation doit permettre à l'architecte d'évaluer
rapidement la pertinence de la solution sans avoir à comprendre en
détail le patron de reconfiguration. Dans l'idée de titre des patrons, on a garder l'idée de la métaphore.

\paragraph{}
Dans la continuité du titre l'intention doit aboutir à la
comprenhension de la métaphore. en expliquant, un approche naive et
ses inconvénients puis on explique l'idée générale du patron et ses
avantages en comparaison avec l'approche naive. 
l'approche du patron  

\subsection{résumé } 

\paragraph{}
A ce stade on a identifié et documenté plusieurs patrons de
reconfiguration. On a donc la 

\begin{itemize}
\item quiescence : dont l'objectif le principe est de rendre passif,
les composants dépendant du composant ciblé par la reconfiguration. 
\item tranquilité : est une variante de la quiescence qui assouplie
les critères de reconfiguration. Elle décrit les conditions dans
lesquelles la reconfiguration pêtre réalisée sans attendre l'état de
tranquilité
\item co-evolution : contrairement à la quiescence et tranquilité, à
déployer directement la nouvelle version du composant ciblé par la
reconfiguration. La conséquence est que les deux versions d'un
composant s'execute simultanément. 
\item opportuniste : par rapport aux autres, il concerne une
granularité différente puisqu'il concerne seulement une opération de
déconnexion et connexion qui est réalisé dès que l'occasion se
présente.  
\end{itemize}




