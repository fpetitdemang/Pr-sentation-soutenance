\section{Patron et processus de reconfiguration}

\subsection{Diapo : Comparaison de critère de réutilisation des
décisions de reconfiguration}

Pour définir des critères de documentation réutilisables, on a fait la
synhtèse des parties descriptives des patrons de conception dans
l'état de l'art. On en en tirer la liste suivante. 

%%%%%%%%%%%%%%%%%%%%%%%%%%%%%%%%%%%%%%%%%%%%%%%%%
%%%\subsection{Diapo : Documentation formelle Allen}
On peut classifier le type de documentation suivant son niveau de
formalité. 

%%%%
\paragraph{}
Si on regarde la proposition de Allen, il contribue à la 
réutilisation en considérant la phase de reconfiguration comme un
aspect à part dans le processus de developpement d'un système et donc
en proposant un langage et des outils dédiés à la reconfiguration
dynamique.

\paragraph{}
\begin{itemize} 
\item Ce qu'on a identifié comme le contexte de la reconfiguraiton est une
description formelle du style architectural. 
\item Ce qu'on a identifié
comme le problème sont les invariants à respecter pendant la
reconfiguration. 
\item La solution est une description des événéments qui déclenchent des
ensemble d'opération de reconfiguration. 
\end{itemize}

\paragraph{}
Ce qui limite la réutilisation de la documentation c'est l'absence
d'une séparation claire entre ce qui est de l'ordre du contexte et de
la problématique, ensuite le niveau de formalisme utilisé est très
formelle ce qui limite la comprension globale de la documentation. 
%

%%%%%%%%%%%%%%%%%%%%%%%%%%%%%%
%%%%\subsection{Diapo : Documentation formelle Oliveira}
%%%%%
\paragraph{} 
Si on regarde en détaille la proposition de Oliveira. 
%
Si on regarde les aspects réutilisations, elle concerne la description
formelle de primitive de reconfiguration récurrente qu'elle a
identifié comme des patrons de reconfiguration.  

\begin{itemize}
\item Le titre des patrons correspond au résultat de l'application du patron
comme la suppression d'un ensemble de canaux de communication. 
\item Le contexte est décrit par la description formelle des
primitives de reconfiguration utilisé dans la solution
\item Le problème correspond à la description de l'archicture
initaiale et ciblé
\item Ensuite la solution décrit les primitives de reconfiguration
appelée
\end{itemize}

\paragraph{}
Ce qui limite la réutilisation est une granularité trop fine qui
caractérise que des cas triviaux de reconfiguration. 

%
%%%%%%%%%%%%%%%%%%%%%%%%%%%%%%%%%%
%%%%\subsection{Diapo : Documentation semi-formelle}
\paragraph{}
D'autre contribution comme Gomaa base la documentation sur des
descriptions semi-formelles. La contribution de réutilisation est la
documentation de solution de quiescence dans une différent contexte
architecturaux. 

\begin{itemize}
\item Le titre du patron capture la contexte d'utilisation du patron., ici
c'est le style architectural dans lequel est mis en \oe{}uvre le
patron. 
\item Le contexte, documente le style architectural en décrivant le
type des composants et la nature de leurs interactions avec les autres
composants avec un diagramme de collaboration. 
\item La solution décrit avec un diagramme d'état modélisant l'effet
de la reconfiguration sur les composants du système. 
\end{itemize}

\paragraph{}
Ce qui limite la réutilisation 
\begin{itemize}
\item Le titre ne permet de choisir un patron en fonction du type de
solution. 
\item Ensuite le choix d'un diagramme de collaboration n'est pas adapté au
contexte sds, puisque la taille des SdSs rend inutilisable ce type
documentation. 
\item La description de la conséquence du patron n'est pas décrit
séparement de la solution. 
\end{itemize}

Par contre on a trouvé pertinent l'utilisation des diagrammes d'état
pour modéliser les événéments de reconfigraiton, les opérations et les
états des compsoants.
 
\subsection{Diapo : Réutilisation des décisions de reconfiguration}

\paragraph{}
Dans un contexte, SdS l'architecte doit avoir une documentation qui
satisfait ts les critères de réutilisation ce qui
permet de prendre en compte l'hétérogénéité et l'évolution de l'environnement
qui est le résultat de l'indépendance opérationnelle et managériale
des systèmes constituants.  

\paragraph{}
Pour cela, on définit explicitement le contenu d'une documentation de
reconfiguration.  

\paragraph{}
De façon classique, un titre et une intention qui
permette à l'architecte de donner l'intuition du principe mis en
\oe{}uvre par la solution documentée. En général, le titre correpond à
une métaphore du principe suivi par la solution et l'intention
explique pourquoi une solution naive ne fonctionne pas et quel est un
principe plus adapté.   

\paragraph{}
pour documenter le contexte de reconfiguration, on s'est intéressé à
la documentation du style architectural par un diagramme de block. Du
point de vue de la reconfiguration, c'est une abstraction qui modélise
les dépendances en tre les composants et les contraintes
d'intérations. 

\paragraph{}
Par rapport à Allen et Oliveira, on a exclu l'utilisation de langage
formel qui n'aide pas à la compréhension. On a suivi un approche semie
formelle avec l'utilisation de langage graphique. Par rapport à Gomaa,
on a exclu l'utilisation de diagramme de collaboration qui ne sont pas
utilisables du fait de la taille des SdSs. 
%L'utilisation d'exemple, permette de préciser les contraintes liées au
%style et le rôle des composant et le type de leur intéraction.  

%%%i
\paragraph{}
Pour la description du problème, on a décrit quelle est le résultat de
l'application de la reconfiguration. 
%
Dans notre documentation, on a documenté le problème par la
description de l'architecture ciblée par la reconfiguration. Cette
architecture est biensure liée à au contexte décrit avnt. On a
documenté la reconfiguration initiale et ciblée par la reconfiguration
par un digramme de block qui modélise le changement. 
%
Par rapport à allen, cet aspect
est absent dans la documentation,
oliveira décrit cet aspect mais avec un niveau de formalisme trop bas
puisqu'elle documente le composant des connexions indépendamment d'un
contexte plus globale qui est celui du des systèmes
constituants qui sont impliqués par la reconfiguration. Par rapport, à
gomaa, c'est une aspect nouveau puisque le problème est implicite à la
solution. 
%
%
Ensuite, on a documenté quelles étaients les invariants liées à la
reconfiguration. de façon informelle.  Par rapport à l'ensemble des
documentations existantes, on a documenté explicitement les forces
(les principes) de la solution pour mieux la comprendre.
%

Pour finir on a documenté les forces qui doivent être mis en \oe{}vre
dans la solution. Ce qui permet à l'architecte de comprendre ensuite
qu'elles sont les principes de la solution. 

\paragraph{}
Pour documenter la solution, on a exclu des réprensations formelles. On
s'est également concentré sur l'utilisation de langage semi-formel. On
a trouvé que l'utilisation de diagramme d'état comme utilisé par gomaa
donne une niveau de détaille suffisant pour documenter une solution de
reconfiguration. On l'accompagne d'une description textuelle qui
définit les états, opérations et événements utilisées par la patron. 

\paragraph{}
Pour finir, on a documenté en plus des autres
type de documentation, les conséquences. Il s'est agit décrire, les
conséquences du patron sur les aspects non fonctionnels du système.
mais aussi l'influence des choix d'implémentation sur la
reconfiguration. et suggérer des implémentations particulière. 

\paragraph{}
Ensuite pour continuer notre approche de reconfiguration, on veut
s'intéresser au processus de reconfiguration que l'architecte a à sa
disposition dans l'état de l'art. 
 


\subsection{Diapo : Patron de reconfiguration - amélioration de la
ré-utilisation}

\paragraph{}
Dans la continuité du titre l'intention doit aboutir à la
comprenhension de la métaphore. en expliquant, un approche naive et
ses inconvénients puis on explique l'idée générale du patron et ses
avantages en comparaison avec l'approche naive. 
l'approche du patron  

\paragraph{}
On a constaté qu'une solution consistait à appliquer des
reconfigurations graduellement. 
 
\paragraph{}
On a expliqué d'abord l'intention de la reconfiguration sur la base
d'un exemple. Ici, on a pris un système de service de secours. Dons
le managanement d'une opération est géré par une codis qui doit
déléguer le management à un autres CS. 

On y explique la solution naive qui est d'attendre la fin des
opérations en cours par les systèmes opérationnels. Dans ce cas, la
reconfiguration a aucune change d'aboutir rapidement puisque le codis
et devrait avoir pour résultat une forte dégradation de service. 

La solution proposé dans le patron est de déployer directement le
nouveau composant. Les systèmes opérationnels terminent leurs
transaction en cours avec le codis puis démarre les nouvelles avec le
poste de commandement mobile. Dans ce cas, les deux systèmes
coexistent de façon concurrente et évolution donc conjointement. 

\subsection{Diapo : Patron de reconfiguration - délimitation du problème}

\paragraph{}
Le problème de reconfiguration se réduit à la description du passage
d'une architecture à une autre. 
On peut cependant, raffiner en différent problème suivant les
solutions dévéloppés et en abstraire les régularités pour définir des
patrons de reconfiguration. 

\paragraph{}
On va essayer de déduire l'ensemble des besoins que cherche à
satisfaire la solution pour capturer le problème exprimer par le
patron. 
On a trouvé qu'un bon moyen est d'adord d'expimer l'architecture
intiale, puis celle ciblée. 
L'architecture est bon support pour la compréhension du problème lié à
qualité de service puisqu'elle modélise les dépendances fonctionnelles
enntre les composants. 

\paragraph{}
Ensuite on propose de documenter les contraintes que doivent satisfaire
la solution. Elles peuvent être structurelles ou comportementales.

\paragraph{}
Les forces doivent expliquer les principes mis \oe{}vre dans la
solution. Cela doit aider l'archticte à comprendre quelles sont les
points importants de la solution.  

%%ù
\paragraph{}
Pour documenter des patrons de reconfiguration on a analysé des solutions
récurrentes. Pour en abstraire les régularités qui vont former une
patron de reconfiguration à documenter. 

\subsection{Diapo : Patron de reconfiguration - rédaction de la solution}

\paragraph{}
Ensuite la solution consiste à décrire les mécanismes d'évolution
et d'une grammaire de reconfiguration. La grammaire de reconfiguration
consiste à  décrire les opérations de reconfiguration et leur ordre
afin de d'execution. Pour documenter la grammaire de reconfiguration
nous nous sommes appuyé sur des diagrammes de collaboration et d'état
ainsi qu'une description textuelle. 
 
\paragraph{}
On a documenté la reconfiguration à partir du schéma ... 
et d'une description textuelle ... qui indique la sémantique des
opérations. 


\subsection{Patrons de reconfiguration supplémentaires} 

\paragraph{}
A ce stade on a identifié et documenté plusieurs patrons de
reconfiguration. On a donc la 

\begin{itemize}
\item quiescence : dont l'objectif le principe est de rendre passif,
les composants dépendant du composant ciblé par la reconfiguration. 
\item tranquilité : est une variante de la quiescence qui assouplie
les critères de reconfiguration. Elle décrit les conditions dans
lesquelles la reconfiguration pêtre réalisée sans attendre l'état de
tranquilité
\item co-evolution : contrairement à la quiescence et tranquilité, à
déployer directement la nouvelle version du composant ciblé par la
reconfiguration. La conséquence est que les deux versions d'un
composant s'execute simultanément. 
\item opportuniste : par rapport aux autres, il concerne une
granularité différente puisqu'il concerne seulement une opération de
déconnexion et connexion qui est réalisé dès que l'occasion se
présente.  
\end{itemize}




