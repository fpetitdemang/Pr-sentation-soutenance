\section{Introduction}

\subsection{Titre}
Je vais présenter ma thèse intitulé "Développement évolutionnaire de
SdS avec une approche par patron de reconfiguration" 

\subsection{Diapo 2 : Les SdSs}
\paragraph{}
%Des exemples de SdS sont par exemple :
%Des systèmes de surveillance d'inondation ou les capteurs collaborent
%momentannement pour calculer des informations environnementales

Les services de secours. Les composants sont les services de secours.
Chacune d'elle possède sa propre indépendance managériale puisqu elle
dispose de ces propres ressources et définit elle même sont
organisation interne. Chaque système possède sa propre indépendance
oéprationnelle car ils peuvent réaliser leur mission indépendamment
des autres. Les pompiers peuvent faire de lutte contre les incendies
et faire du secours à personne sans l'aide des autres. Par contre
l'intéraction avec les autres systèmes peut améliorer la mission de 
chaque service de secours. Par exemple, la police peut sécuriser et
faciliter l'accès des pompiers à un sinistre.

%Internet où les composants sont des routeurs qui implémentent le
%protocole ip et collabore volontairement 

%Le Web ou les composants sont les services web dont la composition
%fournissent d'autres services. 
%\paragraph{}
%Les SdSs sont une
%classe système dont les constituants sont eux mêmes des
%systèmes. Les SdSs héritent des caractéristiques des systèmes
%distribués car
%leur composant sont géographiquement distribué. Mais ce qui les
%différencie de système distribué est que leurs consistuants possèdent
%leur propre
%indépendance managérialle et opérationnelle. La raison de la
%collaboration entre ces systèmes est le comportement emergent produit.
%C'est un comportement qui ne peut pas être réalisé par un seul des
%constituants du SdSs et qui apporte en général un bénéfice à tous les
%participants au SdS.


%\paragraph{}
%Parmis les SdSs, on peut distinguer plusieurs catégories de SdS
%suivant le niveau coercitif sur ses composants à prendre en compte 
%dans leur ingénirie.
%Les réseaux de capteurs malgrès qu'ils peuvent opérer indendamment du
%SdS ont des objectifs de collaboration imposés.

%Les services de secours sont plutôt un SdS consensuel car les
%objectifs du SdS sont décidés de façon consensuel.

%Internet définit des objectifs de routage mais l'implication des
%composants reste volontaire.

%Le web les objectifs ne sont pas idenifier et la collaboration est la
%collaboration est volontaire.

\paragraph{}
%Dans la thèse, on a étudieé plutôt les systèmes consensuels. Sans
%donner un importance particulière à la catégorie de SdS. 
Ce qui nous a
particulierement interessé est la conséquence de l'ensemble de ces
catéristiques qui est le développement évolutionnaire du sds. 


\subsection{Diapo : Le dev. évolutionnaire}
\paragraph{}
On a considéré que le développement évolutionnaire est l'évolution
permanente des buts et fonctions du SdS pendant sont execution. 

\paragraph{} 
La cause du développement évolutionnaire est qu'au moment de la
conception, le SdS ne connaît pas toutes les informations sur
l'environnement d'execution au moment de la conception. 

E.g le sdis 56 n'a pas assez de ressources. E.g le les informations ne
sont pas exacte, le sinistre à mal été évalué depuis le début.  

Les raisons peuvent être est managériale, les consituants évoluent en
changeant eux mêmes est les services fournit dans SdS sont impactés.
Le SdS doit s'accomoder des capacités de ses constituants qui peuvent
changer.  

Des SdSs sont formés momentanement en réaction à un événement dont
toutes les informations ne sont pas connu au moment de la conception
et découvertes seulement pendant la phase d'execution.    


\paragraph{}
Pour toutes ces raisons, le SdS est améné à évoluer régulierement
après la phase de conception et de déploiement. Ce qui nécessite des
outils particuliers pour assister l'architecte du SdS afin de maîtriser au
mieux le développement évolutionnaire. 
  
\subsection{Diapo : reconfiguration dynamique}

\paragraph{}
La reconfiguration dynamique consiste donc à apporter des
modifications sur un système pendant qu'il s'execute. Ces
modifications peuvent être corrective, fonctionnelle si il s'agit
d'ajouter une fonctionnalité, non fonctionnelle si elle modifie la
qualité du système.  La difficulté réside dans la maîtrise des parties
du système interrompue, les dégradations de services, l'intégrité du
système pendant le temps de la reconfiguration. 

\paragraph{}
De manière générale, la reconfiguration dynamique vient du besoin de
pouvoir appliquer des modifications à un système sans le redémarrer
complétement. Dans un contexte SdS, c'est particulierement intéressant
car les SdSs sont typiquement des systèmes qui une fois déployés ne
peuvent pas être arretés puis redemarrés. Les risques seraient suivant
le sds : économique et/ou humaine. 

\paragraph{} 
%
On a pris comme artefact de base pour raisonner sur les reconfigurations
d'un système, ça description en terme de composant et connecteur.
L'intérêt pour la reconfiguration et de pouvoir raisonner sur les
parties modulables du SdS.  cela permet d'identifier les parties
sujettes au changement et de voir leurs dependances dans le SdS.
%
On remarque que les opérations qui composent les scripts de
reconfiguration peuvent être décomposées en opération primitive et
organisées suivant l'importance des modications réalisées sur le SdS. 
Elles sont introspectives quand elle l'opération sert à oberser l'état
du SdS ou intercessives quant elles réalisent des modifications sur le
système. Les opérations sont : 
- ... 
- ...
%
La conception d'une reconfiguration consiste à produire un script
composé de ces opérations primitives qui permettent au système
d'integrer les modifications souhaitées. 

\paragraph{}
A travers la mise en \oe{}uvre
de ces opérations l'architecte cristalisent un certain nombre de
décision de conception qui ont un impacte sur la qualité des services
pendant la reconfiguration et la préservation de la cohérence des
transactions. 
%
Si on regarde un cas typique de stratégie de
reconfiguration qui est la quiescence. L'architecte veut mettre à
jours un composant sans interrompre les parties du système qui ne sont
pas dependantes de ce composant. L'architecte va décider de déconnecter
progressivement les composants qui dépendent du composant à
remplacer. Puis une fois que ts les composants dépendants sont
déconnectés, le composant ciblé par la reconfiguration peut être
remplacé sans rédémarrer complétement le système n'y perdre les
transactions démarrées.


\paragraph{} 
Ce qui peut caractériser un problème de reconfiguration pour les sdss
est : 
\begin{itemize}
\item le contexte de reconfiguration (les opérations de
reconfiguration disponibles) est amené à évoluer. 
\item mais aussi l'hétérogénéité des mécanismes de reconfiguration disponibles au
moment de la reconfiguration. Ce qui diverge des systèmes distribuées
puisque souvent les composants implémentent les mêmes mécanismes de
reconfiguration. 
\end{itemize}
%
Cela introduit de la variabilité également dans les
décisions que peut prendre l'architecte pour réaliser une même
stratégie de reconfiguration. Si on reprend l'exemple de la
quiescence, l'architecte peut décider une approche où le composant
ciblé par la reconfiguration atteind son état de quiescence
naturellement ou alors il peut décider suivant les mécanismes
disponibles de forcer l'état de quiscence pour terminer la
reconfiguration. 

\paragraph{} 
Comme solution à la reconfiguration dynamique nous proposons
d'outiller avec des patrons de reconfiguration et un processus de
reconfiguration pour l'assister face au développement
évolutionnaire des SdSs. 

\subsection{Diapo 6 : Solution - patron de reconfiguration dynamique
et processus de conception}

\paragraph{}
Ce qui nous intéresse est de documenter ce type de décision de
reconfiguration pour la rendre réutilisable. 
%
Il est intéressant de rendre réutilisable. 
Pour adresser un problème de reconfiguration l'architecte doit être
capable de communiquer ses choix de conception dans un but de
validation par ces pairs pour améliorer et valider une
reconfiguration. Ensuite les solutions doivent être capitalisées pour
résoudre plus facilement des problèmes similaires déjà résolus. C'est
un gain de temps et de fiabilité des solutions de conception proposée. 
mêmes erreurs. 

Comme solution on propose la notion de patron de
reconfiguration.

\paragraph{}
Les patrons de reconfiguration ne sont pas utilisables tels quels.
En effet, on pense qu'en pratique l'architecte sera amené à définir
des architectures intermédiaires avant d'appliquer les
reconfigurations initialement décidés. En somme l'application d'un
script de reconfiguration impliquera d'autres reconfigurations. 
% 
nous avons documenté les étapes de conception qui permette
l'élaboration d'une reconfiguration à partir de plusieurs patrons de
reconfiguration. On a donc proposé un processus de conception pour la
conception de reconfiguration dynamique. 
 
%bien connu dans la domaine logiciel. Les patrons de conception sont
%une solution documentée face à un problème récurrent. Les patrons de
%conceptions les plus connus sont ceux du GoF, ils sont utilisés dans
%la programmation orientée objet pour favoriser la réutilisation du
%code. L'exemple est repris dans d'autres champs du logiciel comme la
%conception d'architecture pour intégrer des aspects non fonctionnelles
%comme la scalabilité ou l'adaptabilité. 


