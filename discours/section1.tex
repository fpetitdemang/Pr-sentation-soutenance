\section{Introduction}

\subsection{Titre}
Je vais présenter ma thèse intitulé "Développement évolutionnaire de
SdS avec une approche par patron de reconfiguration" 

\subsection{Diapo 2}
\paragraph{}
Des exemples de SdS sont par exemple :
Des systèmes de surveillance d'inondation ou les capteurs collaborent
momentannement pour calculer des informations environnementales

Les services de secours. Les composants sont les services de secours.
Chacune d'elle possède sa propre indépendance managériale puisqu elle
dispose de ces propres ressources et définit elle même sont
organisation interne. Chaque système possède sa propre indépendance
oéprationnelle car ils peuvent réaliser leur mission indépendamment
des autres. Les pompiers peuvent faire de lutte contre les incendies
et faire du secours à personne sans l'aide des autres. Par contre
l'intéraction avec les autres systèmes peut améliorer la mission de 
chaque service de secours. Par exemple, la police peut sécuriser et
faciliter l'accès des pompiers à un sinistre.

Internet où les composants sont des routeurs qui implémentent le
protocole ip et collabore volontairement 

Le Web ou les composants sont les services web dont la composition
fournissent d'autres services. 
\paragraph{}
Les SdSs sont une
classe système dont les constituants sont eux mêmes des
systèmes. Les SdSs héritent des caractéristiques des systèmes
distribués car
leur composant sont géographiquement distribué. Mais ce qui les
différencie de système distribué est que leurs consistuants possèdent
leur propre
indépendance managérialle et opérationnelle. La raison de la
collaboration entre ces systèmes est le comportement emergent produit.
C'est un comportement qui ne peut pas être réalisé par un seul des
constituants du SdSs et qui apporte en général un bénéfice à tous les
participants au SdS.


\paragraph{}
Parmis les SdSs, on peut distinguer plusieurs catégories de SdS
suivant le niveau coercitif sur ses composants à prendre en compte 
dans leur ingénirie.
Les réseaux de capteurs malgrès qu'ils peuvent opérer indendamment du
SdS ont des objectifs de collaboration imposés.

Les services de secours sont plutôt un SdS consensuel car les
objectifs du SdS sont décidés de façon consensuel.

Internet définit des objectifs de routage mais l'implication des
composants reste volontaire.

Le web les objectifs ne sont pas idenifier et la collaboration est la
collaboration est volontaire.

\paragraph{}
Dans la thèse, on a étudieé plutôt les systèmes consensuels. Sans
donner un importance particulière à la catégorie de SdS. Ce qui nous a
particulierement interessé est la conséquence de l'ensemble de ces
catéristiques qui est le développement évolutionnaire du sds. 


\subsection{Diapo 3}
\paragraph{}
On a considéré que le développement évolutionnaire est l'évolution
permanente des buts et fonctions du SdS pendant sont execution. 

\paragraph{} 
La cause du développement évolutionnaire est qu'au moment de la
conception, le SdS ne connaît pas toutes les informations sur
l'environnement d'execution au moment de la conception. 

Les raisons peuvent être est managériale, les consituants évoluent en
changeant eux mêmes est les services fournit dans SdS sont impactés.
Le SdS doit s'accomoder des capacités de ses constituants qui peuvent
changer. 

Des SdSs sont formés momentanement en réaction à un événement dont
toutes les informations ne sont pas connu au moment de la conception
et découvertes seulement pendant la phase d'execution.    


\paragraph{}
Pour toutes ces raisons, le SdS est améné à évoluer régulierement
après la phase de conception et de déploiement. Ce qui nécessite des
outils particuliers pour assister l'architecte du SdS afin de maîtriser au
mieux le développement évolutionnaire. 
  
\subsection{Diapo 4}

\paragraph{}
Comme solution les SdSs sont amanés à intégrer régulierement de plus
en plus de logiciel pour supporter le developpement évolutionnaire et
en particulier au niveau de la phase de conception de l'architecture
des SdSs.

Dans notre thèse on a pris comme référénce deux
projets européens qui sont DANSE et COMPASS. On peut remarquer la
pré-pondérance du logiciel dans le processus de conception de
l'architecture des SdS.
Les phases de conception intègre des générateur d'architecture, 
des outils de simulation pour la vérification et divers langage pour
modéliser les objectifs du SdS, les contraintes à respecter dans les
architecture et la description du comportement des systèmes
consitituants. 

\paragraph{} 
Un aspect du cycle de vie des systèmes moins étudié pour les SdSs est
la phase de déploiement de la nouvelle architecture dynamiquement et que nous avons
abordé en particulier dans cette thèse à travers le problème de la
reconfiguration dynamique.

\subsection{Diapo 5}

\paragraph{}
La reconfiguration dynamique consiste donc à apporter des
modifications sur un système pendant qu'il s'execute. Ces
modifications peuvent être corrective, fonctionnelle si il s'agit
d'ajouter une fonctionnalité, non fonctionnelle si elle modifie la
qualité du système.  La difficulté réside dans la maîtrise des parties
du système interrompue, les dégradations de services, l'intégrité du
système pendant le temps de la reconfiguration. 

\paragraph{}
De manière générale, la reconfiguration dynamique vient du besoin de
pouvoir appliquer des modifications à un système sans le redémarrer
complétement. Dans un contexte SdS, c'est particulierement intéressant
car les SdSs sont typiquement des systèmes qui une fois déployés ne
peuvent pas être arretés puis redemarrés. Les risques seraient suivant
le sds : économique et/ou humaine. 
 
\paragraph{} 
Comme solution à la reconfiguration dynamique nous proposons
d'outiller l'architecte pour l'assister face au développement
évolutionnaire des SdSs. 

\subsection{Diapo 6}

\paragraph{} 
Pour adresser un problème de reconfiguration l'architecte doit être
capable de communiquer ses choix de conception dans un but de
validation par ces pairs pour améliorer et valider une
reconfiguration. Ensuite les solutions doivent être capitalisées pour
résoudre plus facilement des problèmes similaires déjà résolus. C'est
un gain de temps et de fiabilité des solutions de conception proposée. 
mêmes erreurs. 

\paragraph{}
Comme solution on propose d'intégrer la notion de patron de conception
bien connu dans la domaine logiciel. Les patrons de conception sont
une solution documentée face à un problème récurrent. Les patrons de
conceptions les plus connus sont ceux du GoF, ils sont utilisés dans
la programmation orientée objet pour favoriser la réutilisation du
code. L'exemple est repris dans d'autres champs du logiciel comme la
conception d'architecture pour intégrer des aspects non fonctionnelles
comme la scalabilité ou l'adaptabilité. 

\paragraph{}
Une première difficulté a été l'étude de SdS pour identifier des
problème de reconfiguration que l'on pourrait rencontrer lors de
reconfiguration. Dans l'état de l'art des SdSs peu d'étude de cas se sont intéressées
à capturer des configurations résultantes d'évolutions successives et
des propriétés de reconfiguration souhaitées.

\subsection{Diapo 7}

\paragraph{}
Le challenge à été de capturer des configurations avec le niveau de
précision suffisant pour capturer des propriétés de reconfiguration à
étudier mais aussi avec un niveau d'exactitude correcte pour capturer
des contextes de reconfiguration. 
%\paragraph{} 

\paragraph{}
Une dernière difficulté à été de determiner le processus de
reconfiguration à utiliser pour mettre en \oe{}uvre le concept de patron de
reconfiguration.

\subsection{Diapo 8} 
\paragraph{} 
Dans l'état de l'art on a constaté que la reconfiguration dynamique a
été principalement étudié du point de vue des langages pour la
décrire, des propriétés des protocoles de reconfiguration et
l'automatisation des protocoles de reconfiguration. 
