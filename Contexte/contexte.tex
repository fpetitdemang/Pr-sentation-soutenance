

\begin{frame}{Caractéristique des systèmes de systèmes}
-exemple de SdS,\\
-indep op, managerial, \\
comportement emergent, développement évolutionnaire\\
-dirigé, collaboratif, ...
\end{frame}

\begin{frame}{Cause du développement évolutionnaire}
définition, \\
raison : 
formé momentannement, 
intègre des composants pas prévus à la concepton ,\\
cycle de vie d'un système indépendant du SdS,\\
\end{frame}

\begin{frame}{Pré-pondérance du logiciel dans la conception SdS \& impacte du développement évolutionnaire sur les artefacts logiciels}
sds connu depuis longtemps\\
cycle de vie court\\
logiciel assiste le développement évoluionnaire. 
exemple atelier comme DANSE et COMPASS. (modélisation, génération, analyse)\\
déploiement est une activité récurrente moins prise en compte
\end{frame}


\begin{frame}{Problématique abordée : reconfiguration dynamique}
impossible de 
rédemarrer le SdS,\\
contrôler les propriétés, \\
non anticipé, \\
productivité
\end{frame}

\begin{frame}{Objectif : appliquer principe des patrons de conception logiciels à la reconfiguration dynamique}
définition des patrons logiciels,\\
utilisation des patrons logiciels,\\
utilisation pour la reconfiguration dynamique,\\
documentation réutilisable,\\
utilisation des patrons dans le logiciel,\\
définition d'un processus d'ingénierie.
\end{frame}

\begin{frame}{Problèmes rencontrés : modélisation de configuration}
question de base était appliquer le principe des patrons logiciel à la reconfiguration. \\
besoin de donnée à étudier pour : \\
- chercher à caractériser l'objet étudié dans un contexte sds, \\
- identifier des cas d'étude et les propriétés de reconfiguration \\
manque de cas d'étude complet : \\
comportant des évolutions et les propriétés à préserver 
modèle doivent être précis et exacte, cohérent.
\end{frame}

\begin{frame}{Problèmes recontrés : processus de reconfiguration}
pas de processus claire pour concevoir une 
reconfiguration. \\
état de l'art se focalise sur :\\
- les opérations de bases, \\
- l'automatisation d'étape de reconfigiration, \\
-vérification de reconfiguration\\
\end{frame}

\begin{frame}{Questions de recherche}
Comment l’architecte doit-il procéder pour faire évoluer un système de systèmes après son déploiement, dans le cadre du développement évolutionnaire ?\\
\begin{itemize}
    \item[Q1] Comment l’architecte modélise-t-il la configuration du système avec le niveau de précision et d’exactitude requis ?
    \item[Q2] Quel processus d’ingénierie l’architecte doit-il suivre pour concevoir la reconfiguration d’évolution du système de systèmes ?
    \item[Q3] Comment l’architecte peut-il documenter ses choix de conception d’une reconfiguration ?
\end{itemize}
\end{frame}

\begin{frame}{Plan de la soutenance}
\tableofcontents
\end{frame}