\section{Patron de reconfiguration et Processus de reconfiguration}

\begin{frame}{Mécanismes de base pour la reconfiguration dynamique}
introspection, intercession
\end{frame}

\begin{frame}{Décision de reconfiguration}
    exemple quiescence (intention, contexte, problème, 
solution, force, conséquence)
\end{frame}

\begin{frame}{Caractéristique des décisions de reconfiguration pour les SdSs}
- Critères de reconfiguration liées aux  SdSs : \\
hétérogénieté des mécanismes disponibles, \\
variabilité des choix pour une même solution.\\
besoin de réutilisation pour conception plus rapide\\
- propose d'améliorer la description des patron de reconfiguraiton dynamique     
\end{frame}

\begin{frame}{Type de documentation des décisions : Patron de reconfiguation}
    semi-formelle\\
formelle\\
conclusion : la description du contexte des SdSs est mal pris en compte
\end{frame}

\begin{frame}{Processus d'ingéniere}
- méthode génération automatique, 
langage, ...\\
- comment concevoir un script dans une contexte sds
(variabilité des décisions, hétérogénéité des opérations)
conclusion : - peu de littérature sur le processus de conception de reconfiguraiton en lui-même
- comment maîtriser la dégradation de service \\
- prendre en compte les évolutions de l'architecture pendant la reconfiguration 
(pour la reconfiguration et par l'environnement) ?      
\end{frame}

\begin{frame}{Schéma approche générale}
- utilisation d'un ensemble de patron de reconfiguration pour assister l'architecte dans la définition d'un script de reconfiguration\\
- processus de reconfiguration récursif pour guider l'architecte dans la composition des patrons\\
- définition des sections et contenus, comment construire un patron de reconfiguration? 
\end{frame}

\begin{frame}{Patron de reconfiguration}
    définition des sections
\end{frame}

\begin{frame}{Construction d'un patron de reconfiguration}
    donnée en entrée (bibliographie), 
principe récurrent
\end{frame}

\begin{frame}{Catalogue de patrons de reconfiguration}
    Autres patrons de notre bibliographie (quiescence, tranquilité re-rerouting),\\
patron robuste car abstrait de solution éprouvée
flexibilité de l'implémentation\\
patron se focalise sur des propriétés de qualité de service,\\ 
manque  description des synérgie entre les patrons de reconfiguration
\end{frame}

\begin{frame}{Rappel schéma approche générale}
    
\end{frame}

\begin{frame}{Principe d'une architecture de transition}
\end{frame}

\begin{frame}{Principe du processus itératif}
\end{frame}

\begin{frame}{Utilisation conjointe du processus et des patrons de reconfiguration}
\end{frame}