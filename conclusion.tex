\section{Conclusions et perspectives}

\begin{frame}{Q1 : comment l’architecte modélise-t-il la configuration du système avec le niveau de précision et d’exactitude requis ?
 }
synthèse comparative DANSE et COMPASS\\
proposition d'un framework (UPDM, sysml, ocl)\\
modélisation de 3 configurations (?? diagrammes)
\end{frame}

\begin{frame}{Q2 : comment l’architecte peut-il documenter ses choix de conception d’une reconfiguration ?}
id. critère de réutilisation\\
limites des doc. existantes\\
définition patron de reconfiguration\\
documentation de plusieurs patron.
\end{frame}

\begin{frame}{Q3 : quel processus d’ingénierie l’architecte doit-il
suivre pour concevoir la reconfiguration d’évolution du système de
systèmes ?}
processus iter et récurtent 
\end{frame}

\begin{frame}{Conclusion}
montrer la faisailité de l'approche\\

\end{frame}

\begin{frame}{Perspective}
validation\\
traçabilité des décisions\\
composabilité des patrons\\    
vérification\\
envi. socio-technique
\end{frame}

\begin{frame}[plain]
    Merci de votre attention
\end{frame}
