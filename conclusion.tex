\section{Conclusions et perspectives}



\begin{frame}{Conclusion}

Démontré
\begin{itemize}
\item Limitation des documentations de reconfiguration pour les SdSs
\item Caractère artisanal du processus de reconfiguration
\end{itemize}

Expérimentation 
\begin{itemize}
\item Développement d'un framework de reconfiguration  pour simulation d'une
reconfiguration de SdS
\item Développement d'un framework de modélisation de SdS
\item Modélisation d'une étude de cas 
\end{itemize} 

Apports :
\begin{itemize}
\item Patron de reconfiguration 
\begin{itemize}
\item Définiton d'une documentation de reconfiguration réutilisable
\item Proposition de plusieurs patrons de reconfiguration
\end{itemize}
\item Processus de reconfiguration 
\begin{itemize}
\item Proposition d'un processus de reconfiguration
\item Prise en compte des dégradations de service
\end{itemize}
\end{itemize}
\end{frame}
\begin{frame}{Q1 : comment l’architecte modélise-t-il la configuration du système avec le niveau de précision et d’exactitude requis ?}
Difficulté : 
\begin{itemize}
\item Frontières floues du SdS.\\
\end{itemize}


Approche : 
\begin{itemize}
\item Synthèse comparative des frameworks de modélisation DANSE et COMPASS
\end{itemize}

Proposition d'un framework de modélisation basé sur UPDM et SysML : 
\begin{itemize}
\item Amélioration de l'exactitude par raffinements successifs 
\item Amélioration de la précision ajout de primitives architecturales
en OCL.
\end{itemize} 
Résultat : 
\begin{itemize}
\item Modélisation de 3 configurations
\end{itemize}
\end{frame}

\begin{frame}{Perspectives}
\begin{block}{A court terme :} 
\begin{itemize}
\item Amélioration validation
\end{itemize}
\end{block}

\begin{block}{A moyen terme : }
\begin{itemize}
\item Traçabilité des décisions
\item Composabilité des patrons de reconfiguration
\end{itemize}
\end{block}

\begin{block}{A long terme : }
\begin{itemize}
\item Vérification des patrons de reconfiguration
\item Environnement socio-technique
\end{itemize}
\end{block}
\end{frame}


%\begin{frame}[plain]
 %   Merci de votre attention
%\end{frame}
