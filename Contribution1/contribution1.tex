\section{Framework pour la modélisation  des SdSs}

\begin{frame}{Concepts et Objectifs de la modélisation}
- architecture, style architecturale,
comportement, abstraction, 
vue point de vue, exactitude, précision ...\\
- améliore la qualité des systèmes et minimise les efforts/coût de conception\\
(vérification), automatisation génération de modèle\\
- type de langage (graphique),
\end{frame}

\begin{frame}{SySML}
- élément modélisé, mécanismes disponilbles\\
- description des connectivités itéressante pour les 
SdSs
\end{frame}

\begin{frame}{UPDM}
- organisation en vue et point de vue\\
-définition de type de composant/connecteur
\end{frame}

\begin{frame}{DANSE}
-modélisation (selection des vues et 
choix des diagrammes),\\
méthode de vérification (preuve, simulation),\\
précision (propriétés vérifiables) quels sont les modèles utilie \\
exactitude    
\end{frame}

\begin{frame}{Comparaison de DANSE et COMPASS}
- objectif de modélisation des vues pareil\\
- utilisation des diagrammes différentes pour modéliser les objetctifs d'interaction\\
les points négatifs : \\
- pas de vérification de l'exactide. raffinement des propriétés dans des langages formelles\\
\end{frame}

\begin{frame}{Cas d'étude}
mission et systèmes\\
caractéristiques SdS\\
architectures identifiées\\
\end{frame}

\begin{frame}{Stratégie de modélisation}
- précision (propriétés à vérifier pendnat la reconfi)?\\ 
- exactitude \\
- modélisation (selection des vues et 
choix des diagrammes),
méthode d'analyse,    
\end{frame}

\begin{frame}{Choix de modélisation}
propriétés de reconfiguration vérifiée?\\
choix des vues et modèles
(selection des vues parmie???)\\
raffinement de la sémantique oéprationnelle (pi-calcul), \\
contrainte ocl
\end{frame}

\begin{frame}{Vérification de l'éxactitude}
choix du processus,\\
approche descendante, \\
favorise la vérification manuelle avec l'utilisation 
de primitive architecturale
\end{frame}

\begin{frame}{Modélisation des architectures}
tailles des modèes : nombre de diagrammes? composants/connecteur décrit?\\
exemples des diagrammes, utilisation des contraintes ocls
\end{frame}

\begin{frame}{Résumé}
point fort/faible,\\
- (minimiser les efforts de modélisation avec ocl, 
utilisation de primitive archi permet de vérifier l'eaxctitude des propriétés)\\
- méthode de validation (qualitatif : on a capturer les propriétés que l'on veut vérifier pendant la reconf.)\\
- les modèles peuvent être plus détaillés\\
\end{frame}